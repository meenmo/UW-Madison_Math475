% --------------------------------------------------------------
% This is all preamble stuff that you don't have to worry about.
% Head down to where it says "Start here"
% --------------------------------------------------------------
 
\documentclass[12pt]{article}
 
        
     
 

\usepackage[margin=1in]{geometry} 
\usepackage{amsmath,amsthm,amssymb,scrextend}
\usepackage{fancyhdr}
\usepackage{enumitem}
\usepackage{amsmath}
\usepackage{amssymb}
\usepackage{textcomp}
\usepackage{fancybox}
\usepackage{tikz}
\usepackage{tasks}
\usepackage{amsmath,thmtools}
\usepackage[T1]{fontenc}
\usepackage{mathpazo}
\usepackage{graphicx}
\pagestyle{fancy}
\usepackage{adjustbox} % Used to constrain images to a maximum size 
\usepackage{xcolor} % Allow colors to be defined
\usepackage{geometry} % Used to adjust the document margins
\usepackage{textcomp}

\AtBeginDocument{%
        \def\PYZsq{\textquotesingle}% Upright quotes in Pygmentized code
    }
\usepackage{upquote} % Upright quotes for verbatim code
\usepackage{eurosym} % defines \euro
\usepackage[mathletters]{ucs} % Extended unicode (utf-8) support
\usepackage[utf8x]{inputenc} % Allow utf-8 characters in the tex document
\usepackage{fancyvrb} % verbatim replacement that allows latex
\usepackage{grffile} % extends the file name processing of package graphics 
                     % to support a larger range 
% The hyperref package gives us a pdf with properly built
% internal navigation ('pdf bookmarks' for the table of contents,
% internal cross-reference links, web links for URLs, etc.)
\usepackage{hyperref}
\usepackage{longtable} % longtable support required by pandoc >1.10
\usepackage{booktabs}  % table support for pandoc > 1.12.2
\usepackage[inline]{enumitem} % IRkernel/repr support (it uses the enumerate* environment)
\usepackage[normalem]{ulem} % ulem is needed to support strikethroughs (\sout)
                            % normalem makes italics be italics, not underlines
\usepackage{mathrsfs}


% --------------------------------------------------------------
%                        Jupyter Notebook
% --------------------------------------------------------------
% Colors for the hyperref package
    \definecolor{urlcolor}{rgb}{0,.145,.698}
    \definecolor{linkcolor}{rgb}{.71,0.21,0.01}
    \definecolor{citecolor}{rgb}{.12,.54,.11}

    % ANSI colors
    \definecolor{ansi-black}{HTML}{3E424D}
    \definecolor{ansi-black-intense}{HTML}{282C36}
    \definecolor{ansi-red}{HTML}{E75C58}
    \definecolor{ansi-red-intense}{HTML}{B22B31}
    \definecolor{ansi-green}{HTML}{00A250}
    \definecolor{ansi-green-intense}{HTML}{007427}
    \definecolor{ansi-yellow}{HTML}{DDB62B}
    \definecolor{ansi-yellow-intense}{HTML}{B27D12}
    \definecolor{ansi-blue}{HTML}{208FFB}
    \definecolor{ansi-blue-intense}{HTML}{0065CA}
    \definecolor{ansi-magenta}{HTML}{D160C4}
    \definecolor{ansi-magenta-intense}{HTML}{A03196}
    \definecolor{ansi-cyan}{HTML}{60C6C8}
    \definecolor{ansi-cyan-intense}{HTML}{258F8F}
    \definecolor{ansi-white}{HTML}{C5C1B4}
    \definecolor{ansi-white-intense}{HTML}{A1A6B2}
    \definecolor{ansi-default-inverse-fg}{HTML}{FFFFFF}
    \definecolor{ansi-default-inverse-bg}{HTML}{000000}

    % commands and environments needed by pandoc snippets
    % extracted from the output of `pandoc -s`
    \providecommand{\tightlist}{%
      \setlength{\itemsep}{0pt}\setlength{\parskip}{0pt}}
    \DefineVerbatimEnvironment{Highlighting}{Verbatim}{commandchars=\\\{\}}
    % Add ',fontsize=\small' for more characters per line
    \newenvironment{Shaded}{}{}
    \newcommand{\KeywordTok}[1]{\textcolor[rgb]{0.00,0.44,0.13}{\textbf{{#1}}}}
    \newcommand{\DataTypeTok}[1]{\textcolor[rgb]{0.56,0.13,0.00}{{#1}}}
    \newcommand{\DecValTok}[1]{\textcolor[rgb]{0.25,0.63,0.44}{{#1}}}
    \newcommand{\BaseNTok}[1]{\textcolor[rgb]{0.25,0.63,0.44}{{#1}}}
    \newcommand{\FloatTok}[1]{\textcolor[rgb]{0.25,0.63,0.44}{{#1}}}
    \newcommand{\CharTok}[1]{\textcolor[rgb]{0.25,0.44,0.63}{{#1}}}
    \newcommand{\StringTok}[1]{\textcolor[rgb]{0.25,0.44,0.63}{{#1}}}
    \newcommand{\CommentTok}[1]{\textcolor[rgb]{0.38,0.63,0.69}{\textit{{#1}}}}
    \newcommand{\OtherTok}[1]{\textcolor[rgb]{0.00,0.44,0.13}{{#1}}}
    \newcommand{\AlertTok}[1]{\textcolor[rgb]{1.00,0.00,0.00}{\textbf{{#1}}}}
    \newcommand{\FunctionTok}[1]{\textcolor[rgb]{0.02,0.16,0.49}{{#1}}}
    \newcommand{\RegionMarkerTok}[1]{{#1}}
    \newcommand{\ErrorTok}[1]{\textcolor[rgb]{1.00,0.00,0.00}{\textbf{{#1}}}}
    \newcommand{\NormalTok}[1]{{#1}}
    
    % Additional commands for more recent versions of Pandoc
    \newcommand{\ConstantTok}[1]{\textcolor[rgb]{0.53,0.00,0.00}{{#1}}}
    \newcommand{\SpecialCharTok}[1]{\textcolor[rgb]{0.25,0.44,0.63}{{#1}}}
    \newcommand{\VerbatimStringTok}[1]{\textcolor[rgb]{0.25,0.44,0.63}{{#1}}}
    \newcommand{\SpecialStringTok}[1]{\textcolor[rgb]{0.73,0.40,0.53}{{#1}}}
    \newcommand{\ImportTok}[1]{{#1}}
    \newcommand{\DocumentationTok}[1]{\textcolor[rgb]{0.73,0.13,0.13}{\textit{{#1}}}}
    \newcommand{\AnnotationTok}[1]{\textcolor[rgb]{0.38,0.63,0.69}{\textbf{\textit{{#1}}}}}
    \newcommand{\CommentVarTok}[1]{\textcolor[rgb]{0.38,0.63,0.69}{\textbf{\textit{{#1}}}}}
    \newcommand{\VariableTok}[1]{\textcolor[rgb]{0.10,0.09,0.49}{{#1}}}
    \newcommand{\ControlFlowTok}[1]{\textcolor[rgb]{0.00,0.44,0.13}{\textbf{{#1}}}}
    \newcommand{\OperatorTok}[1]{\textcolor[rgb]{0.40,0.40,0.40}{{#1}}}
    \newcommand{\BuiltInTok}[1]{{#1}}
    \newcommand{\ExtensionTok}[1]{{#1}}
    \newcommand{\PreprocessorTok}[1]{\textcolor[rgb]{0.74,0.48,0.00}{{#1}}}
    \newcommand{\AttributeTok}[1]{\textcolor[rgb]{0.49,0.56,0.16}{{#1}}}
    \newcommand{\InformationTok}[1]{\textcolor[rgb]{0.38,0.63,0.69}{\textbf{\textit{{#1}}}}}
    \newcommand{\WarningTok}[1]{\textcolor[rgb]{0.38,0.63,0.69}{\textbf{\textit{{#1}}}}}
    
    
    % Pygments definitions
    
\makeatletter
\def\PY@reset{\let\PY@it=\relax \let\PY@bf=\relax%
    \let\PY@ul=\relax \let\PY@tc=\relax%
    \let\PY@bc=\relax \let\PY@ff=\relax}
\def\PY@tok#1{\csname PY@tok@#1\endcsname}
\def\PY@toks#1+{\ifx\relax#1\empty\else%
    \PY@tok{#1}\expandafter\PY@toks\fi}
\def\PY@do#1{\PY@bc{\PY@tc{\PY@ul{%
    \PY@it{\PY@bf{\PY@ff{#1}}}}}}}
\def\PY#1#2{\PY@reset\PY@toks#1+\relax+\PY@do{#2}}

\expandafter\def\csname PY@tok@w\endcsname{\def\PY@tc##1{\textcolor[rgb]{0.73,0.73,0.73}{##1}}}
\expandafter\def\csname PY@tok@c\endcsname{\let\PY@it=\textit\def\PY@tc##1{\textcolor[rgb]{0.25,0.50,0.50}{##1}}}
\expandafter\def\csname PY@tok@cp\endcsname{\def\PY@tc##1{\textcolor[rgb]{0.74,0.48,0.00}{##1}}}
\expandafter\def\csname PY@tok@k\endcsname{\let\PY@bf=\textbf\def\PY@tc##1{\textcolor[rgb]{0.00,0.50,0.00}{##1}}}
\expandafter\def\csname PY@tok@kp\endcsname{\def\PY@tc##1{\textcolor[rgb]{0.00,0.50,0.00}{##1}}}
\expandafter\def\csname PY@tok@kt\endcsname{\def\PY@tc##1{\textcolor[rgb]{0.69,0.00,0.25}{##1}}}
\expandafter\def\csname PY@tok@o\endcsname{\def\PY@tc##1{\textcolor[rgb]{0.40,0.40,0.40}{##1}}}
\expandafter\def\csname PY@tok@ow\endcsname{\let\PY@bf=\textbf\def\PY@tc##1{\textcolor[rgb]{0.67,0.13,1.00}{##1}}}
\expandafter\def\csname PY@tok@nb\endcsname{\def\PY@tc##1{\textcolor[rgb]{0.00,0.50,0.00}{##1}}}
\expandafter\def\csname PY@tok@nf\endcsname{\def\PY@tc##1{\textcolor[rgb]{0.00,0.00,1.00}{##1}}}
\expandafter\def\csname PY@tok@nc\endcsname{\let\PY@bf=\textbf\def\PY@tc##1{\textcolor[rgb]{0.00,0.00,1.00}{##1}}}
\expandafter\def\csname PY@tok@nn\endcsname{\let\PY@bf=\textbf\def\PY@tc##1{\textcolor[rgb]{0.00,0.00,1.00}{##1}}}
\expandafter\def\csname PY@tok@ne\endcsname{\let\PY@bf=\textbf\def\PY@tc##1{\textcolor[rgb]{0.82,0.25,0.23}{##1}}}
\expandafter\def\csname PY@tok@nv\endcsname{\def\PY@tc##1{\textcolor[rgb]{0.10,0.09,0.49}{##1}}}
\expandafter\def\csname PY@tok@no\endcsname{\def\PY@tc##1{\textcolor[rgb]{0.53,0.00,0.00}{##1}}}
\expandafter\def\csname PY@tok@nl\endcsname{\def\PY@tc##1{\textcolor[rgb]{0.63,0.63,0.00}{##1}}}
\expandafter\def\csname PY@tok@ni\endcsname{\let\PY@bf=\textbf\def\PY@tc##1{\textcolor[rgb]{0.60,0.60,0.60}{##1}}}
\expandafter\def\csname PY@tok@na\endcsname{\def\PY@tc##1{\textcolor[rgb]{0.49,0.56,0.16}{##1}}}
\expandafter\def\csname PY@tok@nt\endcsname{\let\PY@bf=\textbf\def\PY@tc##1{\textcolor[rgb]{0.00,0.50,0.00}{##1}}}
\expandafter\def\csname PY@tok@nd\endcsname{\def\PY@tc##1{\textcolor[rgb]{0.67,0.13,1.00}{##1}}}
\expandafter\def\csname PY@tok@s\endcsname{\def\PY@tc##1{\textcolor[rgb]{0.73,0.13,0.13}{##1}}}
\expandafter\def\csname PY@tok@sd\endcsname{\let\PY@it=\textit\def\PY@tc##1{\textcolor[rgb]{0.73,0.13,0.13}{##1}}}
\expandafter\def\csname PY@tok@si\endcsname{\let\PY@bf=\textbf\def\PY@tc##1{\textcolor[rgb]{0.73,0.40,0.53}{##1}}}
\expandafter\def\csname PY@tok@se\endcsname{\let\PY@bf=\textbf\def\PY@tc##1{\textcolor[rgb]{0.73,0.40,0.13}{##1}}}
\expandafter\def\csname PY@tok@sr\endcsname{\def\PY@tc##1{\textcolor[rgb]{0.73,0.40,0.53}{##1}}}
\expandafter\def\csname PY@tok@ss\endcsname{\def\PY@tc##1{\textcolor[rgb]{0.10,0.09,0.49}{##1}}}
\expandafter\def\csname PY@tok@sx\endcsname{\def\PY@tc##1{\textcolor[rgb]{0.00,0.50,0.00}{##1}}}
\expandafter\def\csname PY@tok@m\endcsname{\def\PY@tc##1{\textcolor[rgb]{0.40,0.40,0.40}{##1}}}
\expandafter\def\csname PY@tok@gh\endcsname{\let\PY@bf=\textbf\def\PY@tc##1{\textcolor[rgb]{0.00,0.00,0.50}{##1}}}
\expandafter\def\csname PY@tok@gu\endcsname{\let\PY@bf=\textbf\def\PY@tc##1{\textcolor[rgb]{0.50,0.00,0.50}{##1}}}
\expandafter\def\csname PY@tok@gd\endcsname{\def\PY@tc##1{\textcolor[rgb]{0.63,0.00,0.00}{##1}}}
\expandafter\def\csname PY@tok@gi\endcsname{\def\PY@tc##1{\textcolor[rgb]{0.00,0.63,0.00}{##1}}}
\expandafter\def\csname PY@tok@gr\endcsname{\def\PY@tc##1{\textcolor[rgb]{1.00,0.00,0.00}{##1}}}
\expandafter\def\csname PY@tok@ge\endcsname{\let\PY@it=\textit}
\expandafter\def\csname PY@tok@gs\endcsname{\let\PY@bf=\textbf}
\expandafter\def\csname PY@tok@gp\endcsname{\let\PY@bf=\textbf\def\PY@tc##1{\textcolor[rgb]{0.00,0.00,0.50}{##1}}}
\expandafter\def\csname PY@tok@go\endcsname{\def\PY@tc##1{\textcolor[rgb]{0.53,0.53,0.53}{##1}}}
\expandafter\def\csname PY@tok@gt\endcsname{\def\PY@tc##1{\textcolor[rgb]{0.00,0.27,0.87}{##1}}}
\expandafter\def\csname PY@tok@err\endcsname{\def\PY@bc##1{\setlength{\fboxsep}{0pt}\fcolorbox[rgb]{1.00,0.00,0.00}{1,1,1}{\strut ##1}}}
\expandafter\def\csname PY@tok@kc\endcsname{\let\PY@bf=\textbf\def\PY@tc##1{\textcolor[rgb]{0.00,0.50,0.00}{##1}}}
\expandafter\def\csname PY@tok@kd\endcsname{\let\PY@bf=\textbf\def\PY@tc##1{\textcolor[rgb]{0.00,0.50,0.00}{##1}}}
\expandafter\def\csname PY@tok@kn\endcsname{\let\PY@bf=\textbf\def\PY@tc##1{\textcolor[rgb]{0.00,0.50,0.00}{##1}}}
\expandafter\def\csname PY@tok@kr\endcsname{\let\PY@bf=\textbf\def\PY@tc##1{\textcolor[rgb]{0.00,0.50,0.00}{##1}}}
\expandafter\def\csname PY@tok@bp\endcsname{\def\PY@tc##1{\textcolor[rgb]{0.00,0.50,0.00}{##1}}}
\expandafter\def\csname PY@tok@fm\endcsname{\def\PY@tc##1{\textcolor[rgb]{0.00,0.00,1.00}{##1}}}
\expandafter\def\csname PY@tok@vc\endcsname{\def\PY@tc##1{\textcolor[rgb]{0.10,0.09,0.49}{##1}}}
\expandafter\def\csname PY@tok@vg\endcsname{\def\PY@tc##1{\textcolor[rgb]{0.10,0.09,0.49}{##1}}}
\expandafter\def\csname PY@tok@vi\endcsname{\def\PY@tc##1{\textcolor[rgb]{0.10,0.09,0.49}{##1}}}
\expandafter\def\csname PY@tok@vm\endcsname{\def\PY@tc##1{\textcolor[rgb]{0.10,0.09,0.49}{##1}}}
\expandafter\def\csname PY@tok@sa\endcsname{\def\PY@tc##1{\textcolor[rgb]{0.73,0.13,0.13}{##1}}}
\expandafter\def\csname PY@tok@sb\endcsname{\def\PY@tc##1{\textcolor[rgb]{0.73,0.13,0.13}{##1}}}
\expandafter\def\csname PY@tok@sc\endcsname{\def\PY@tc##1{\textcolor[rgb]{0.73,0.13,0.13}{##1}}}
\expandafter\def\csname PY@tok@dl\endcsname{\def\PY@tc##1{\textcolor[rgb]{0.73,0.13,0.13}{##1}}}
\expandafter\def\csname PY@tok@s2\endcsname{\def\PY@tc##1{\textcolor[rgb]{0.73,0.13,0.13}{##1}}}
\expandafter\def\csname PY@tok@sh\endcsname{\def\PY@tc##1{\textcolor[rgb]{0.73,0.13,0.13}{##1}}}
\expandafter\def\csname PY@tok@s1\endcsname{\def\PY@tc##1{\textcolor[rgb]{0.73,0.13,0.13}{##1}}}
\expandafter\def\csname PY@tok@mb\endcsname{\def\PY@tc##1{\textcolor[rgb]{0.40,0.40,0.40}{##1}}}
\expandafter\def\csname PY@tok@mf\endcsname{\def\PY@tc##1{\textcolor[rgb]{0.40,0.40,0.40}{##1}}}
\expandafter\def\csname PY@tok@mh\endcsname{\def\PY@tc##1{\textcolor[rgb]{0.40,0.40,0.40}{##1}}}
\expandafter\def\csname PY@tok@mi\endcsname{\def\PY@tc##1{\textcolor[rgb]{0.40,0.40,0.40}{##1}}}
\expandafter\def\csname PY@tok@il\endcsname{\def\PY@tc##1{\textcolor[rgb]{0.40,0.40,0.40}{##1}}}
\expandafter\def\csname PY@tok@mo\endcsname{\def\PY@tc##1{\textcolor[rgb]{0.40,0.40,0.40}{##1}}}
\expandafter\def\csname PY@tok@ch\endcsname{\let\PY@it=\textit\def\PY@tc##1{\textcolor[rgb]{0.25,0.50,0.50}{##1}}}
\expandafter\def\csname PY@tok@cm\endcsname{\let\PY@it=\textit\def\PY@tc##1{\textcolor[rgb]{0.25,0.50,0.50}{##1}}}
\expandafter\def\csname PY@tok@cpf\endcsname{\let\PY@it=\textit\def\PY@tc##1{\textcolor[rgb]{0.25,0.50,0.50}{##1}}}
\expandafter\def\csname PY@tok@c1\endcsname{\let\PY@it=\textit\def\PY@tc##1{\textcolor[rgb]{0.25,0.50,0.50}{##1}}}
\expandafter\def\csname PY@tok@cs\endcsname{\let\PY@it=\textit\def\PY@tc##1{\textcolor[rgb]{0.25,0.50,0.50}{##1}}}

\def\PYZbs{\char`\\}
\def\PYZus{\char`\_}
\def\PYZob{\char`\{}
\def\PYZcb{\char`\}}
\def\PYZca{\char`\^}
\def\PYZam{\char`\&}
\def\PYZlt{\char`\<}
\def\PYZgt{\char`\>}
\def\PYZsh{\char`\#}
\def\PYZpc{\char`\%}
\def\PYZdl{\char`\$}
\def\PYZhy{\char`\-}
\def\PYZsq{\char`\'}
\def\PYZdq{\char`\"}
\def\PYZti{\char`\~}
% for compatibility with earlier versions
\def\PYZat{@}
\def\PYZlb{[}
\def\PYZrb{]}
\makeatother


    % Exact colors from NB
    \definecolor{incolor}{rgb}{0.0, 0.0, 0.5}
    \definecolor{outcolor}{rgb}{0.545, 0.0, 0.0}



    
    % Prevent overflowing lines due to hard-to-break entities
    \sloppy 
    % Setup hyperref package
    \hypersetup{
      breaklinks=true,  % so long urls are correctly broken across lines
      colorlinks=true,
      urlcolor=urlcolor,
      linkcolor=linkcolor,
      citecolor=citecolor,
      }
    % Slightly bigger margins than the latex defaults
    
    \geometry{verbose,tmargin=1in,bmargin=1in,lmargin=1in,rmargin=1in}


% --------------------------------------------------------------
%                        Jupyter Notebook
% --------------------------------------------------------------
\setlength{\parindent}{0pt}

\newcommand{\N}{\mathbb{N}}
\newcommand{\Z}{\mathbb{Z}}
\newcommand{\I}{\mathbb{I}}
\newcommand{\R}{\mathbb{R}}
\newcommand{\Q}{\mathbb{Q}}
\renewcommand{\qed}{\hfill$\blacksquare$}
\let\newproof\proof
\renewenvironment{proof}{\begin{addmargin}[1em]{0em}\begin{newproof}}{\end{newproof}\end{addmargin}\qed}
% \newcommand{\expl}[1]{\text{\hfill[#1]}$}
\setlength{\parindent}{0pt} 
\newenvironment{theorem}[2][Theorem]{\begin{trivlist}
\item[\hskip \labelsep {\bfseries #1}\hskip \labelsep {\bfseries #2.}]}{\end{trivlist}}
\newenvironment{lemma}[2][Lemma]{\begin{trivlist}
\item[\hskip \labelsep {\bfseries #1}\hskip \labelsep {\bfseries #2.}]}{\end{trivlist}}
\newenvironment{problem}[2][Problem]{\begin{trivlist}
\item[\hskip \labelsep {\bfseries #1}\hskip \labelsep {\bfseries #2.}]}{\end{trivlist}}
\newenvironment{exercise}[2][Exercise]{\begin{trivlist}
\item[\hskip \labelsep {\bfseries #1}\hskip \labelsep {\bfseries #2.}]}{\end{trivlist}}
\newenvironment{reflection}[2][Reflection]{\begin{trivlist}
\item[\hskip \labelsep {\bfseries #1}\hskip \labelsep {\bfseries #2.}]}{\end{trivlist}}
\newenvironment{proposition}[2][Proposition]{\begin{trivlist}
\item[\hskip \labelsep {\bfseries #1}\hskip \labelsep {\bfseries #2.}]}{\end{trivlist}}
\newenvironment{corollary}[2][Corollary]{\begin{trivlist}
\item[\hskip \labelsep {\bfseries #1}\hskip \labelsep {\bfseries #2.}]}{\end{trivlist}}
 

\begin{document}

% --------------------------------------------------------------
%                         Start here
% --------------------------------------------------------------

\lhead{Math 475}
\chead{Homework 6}
\rhead{Meenmo Kang}

\begin{enumerate}
    \item[\bf 6.7.2] Find the number of integers between 1 and 10,000 inclusive that are not divisible by 4, 6, 7, or 10.\\

    Let $S$ be the set of all integers between 1 and 10,000 inclusive, $A_1$ be the subset of numbers divisible by 4, $A_2$ be the subset of numbers divisible by 6, $A_3$ be the subset of numbers divisible by 7, and $A_4$ be the subset of numbers divisible by 10,

    \begin{align*}
        |\overline{A_1} \cap \overline{A_2} \cap \overline{A_3} \cap \overline{A_4}| &= |S| - |A_1| - |A_2| - |A_3| - |A_4| + |A_1 \cap A_2| + |A_1 \cap A_3| + \\
        &\quad\;|A_1\cap A_4| + |A_2 \cap A_3| + |A_2\cap A_4| + |A_3\cap A_4| - \\
        &\quad\; |A_1\cap A_2\cap A_3| - |A_1\cap A_2 \cap A_4| - |A_1\cap A_3\cap A_4| -  \\
        &\quad\; |A_2\cap A_3\cap A_4| + |A_1\cap A_2\cap A_3\cap A_4|
    \end{align*}

    \begin{itemize}
        \item $|S| = 10000$
        \item All Singletons
        \begin{itemize}
            \item $|A_1| = 2500$
            \item $|A_2| = \lfloor \frac{10000}{6}\rfloor = 1666$
            \item $|A_2| = \lfloor \frac{10000}{7}\rfloor = 1428$
            \item $|A_3| = 1000$
            \item $\sum = 6594$\\
        \end{itemize}

        \item All Pairs
        \begin{itemize}
            \item $|A_1\cap A_2| = \lfloor\frac{10000}{12}\rfloor=833$
            \item $|A_1\cap A_3| = \lfloor\frac{10000}{28}\rfloor=357$
            \item $|A_1\cap A_4| = \lfloor\frac{10000}{20}\rfloor=500$
            \item $|A_2\cap A_3| = \lfloor\frac{10000}{42}\rfloor=238$
            \item $|A_2\cap A_4| = \lfloor\frac{10000}{30}\rfloor=333$
            \item $|A_3\cap A_4| = \lfloor\frac{10000}{70}\rfloor=142$
            \item $\sum = 2403$\\
        \end{itemize}

    \item All Triples
    \begin{itemize}
        \item $|A_1\cap A_2\cap A_3| = \lfloor \frac{10000}{84}\rfloor = 119$
        \item $|A_1\cap A_2\cap A_4| = \lfloor\frac{10000}{60}\rfloor = 166$
        \item $|A_1\cap A_3\cap A_4| = \lfloor\frac{10000}{140}\rfloor = 71$
        \item $|A_2\cap A_3\cap A_4| = \lfloor\frac{10000}{210}\rfloor = 47$
        \item $\sum = 403$
    \end{itemize}

    \newpage
    \item All Quadruples
    \begin{itemize}
        \item $|A_1\cap A_2\cap A_3 \cap A_4| = \lfloor\frac{10000}{420}\rfloor = 23$
    \end{itemize}
    \end{itemize}

    $|\overline{A_1} \cap \overline{A_2} \cap \overline{A_3} \cap \overline{A_4}| = 10000 - 6594 + 2403 - 403 + 23 = 5429$
    
    \vspace{1.7\baselineskip}
    Double Check with Python
\begin{Verbatim}[commandchars=\\\{\}]
\PY{n}{count} \PY{o}{=} \PY{l+m+mi}{0}
\PY{k}{for} \PY{n}{i} \PY{o+ow}{in} \PY{n+nb}{range}\PY{p}{(}\PY{l+m+mi}{1}\PY{p}{,}\PY{l+m+mi}{10001}\PY{p}{)}\PY{p}{:}
    \PY{k}{if} \PY{p}{(}\PY{n}{i} \PY{o}{\PYZpc{}} \PY{l+m+mi}{4} \PY{o}{==} \PY{l+m+mi}{0}\PY{p}{)} \PY{o+ow}{or} \PY{p}{(}\PY{n}{i} \PY{o}{\PYZpc{}} \PY{l+m+mi}{6} \PY{o}{==} \PY{l+m+mi}{0}\PY{p}{)} \PY{o+ow}{or}  \PY{p}{(}\PY{n}{i} \PY{o}{\PYZpc{}} \PY{l+m+mi}{7} \PY{o}{==} \PY{l+m+mi}{0}\PY{p}{)} \PY{o+ow}{or} \PY{p}{(}\PY{n}{i} \PY{o}{\PYZpc{}} \PY{l+m+mi}{10} \PY{o}{==} \PY{l+m+mi}{0}\PY{p}{)}\PY{p}{:}
        \PY{n}{count} \PY{o}{+}\PY{o}{=} \PY{l+m+mi}{1}

\PY{n+nb}{print}\PY{p}{(}\PY{l+m+mi}{10000}\PY{o}{\PYZhy{}}\PY{n}{count}\PY{p}{)}
\end{Verbatim}

\begin{Verbatim}[commandchars=\\\{\}]
{\bf 5429}
\end{Verbatim}
    
    \vspace{1.5\baselineskip}
    \item[\bf 6.7.3] Find the number of integers between 1 and 10,000 that are neither perfect squares nor perfect cubes.\\

    \begin{itemize}
        \item Squares
        \begin{itemize}
            \item $1^2 = 1$
            \item $2^2 = 2$
            \item[] $\vdots$
            \item $100^2 = 10000$\\
        \end{itemize}

        \item Cubes
        \begin{itemize}
            \item $1^3 = 1$
            \item $2^3 = 8$
            \item[] $\vdots$
            \item $\lfloor 10000^{1/3} \rfloor = 21$\\
        \end{itemize}

        \item Both
        \begin{itemize}
            \item $1^6 = 1$
            \item $2^6 = 64$
            \item[] $\vdots$
            \item $\lfloor 10000^{1/6}\rfloor = 4$\\
        \end{itemize}

        \item 10000 - 100 - 21 + 4 = 9883
    \end{itemize}

    \item[\bf 6.7.5] Determine the number of 10-combinations of the multiset
    $$S = \{ \infty\cdot a,\;4\cdot b,\; 5\cdot c,\; 7\cdot d\}$$
    
   Since we deal with 10-combinations, consider a set $$S' = \{10\cdot a,\; 10\cdot b,\; 10\cdot c,\; 10\cdot d\}$$
    
    By the Theorem 2.5.1, 
    $$|S'| = \binom{10 + 4-1}{10} = 286$$
    
    My strategy is to subtract improper cases from $|S'|$. Consider cases where
    \begin{itemize}
        \item $b$ is selected more than 4 times. Since we assume at least 5 $b$'s are chosen already, we obtain
        $$\binom{5+4-1}{5} = 56$$
        \item $c$ is selected more than 5 times. Since we assume at least 6 $c$'s are chosen already, we obtain
        $$\binom{4+4-1}{4} = 35$$
        \item $d$ is selected more than 7 times. Since we assume at least 8 $d$'s are chosen already, we obtain
        $$\binom{2+4-1}{2} = 10$$\\
    \end{itemize}
    
    Therefore, the number of 10-combinations of the multiset $S$ is 286 - 56 - 35- 10 = 185
    
    \vspace{2\baselineskip}
    \item[\bf 6.7.6] A bakery sells chocolate, cinnamon, and plain doughnuts and at a particular time has 6 chocolate, 6 cinnamon, and 3 plain. If a box contains 12 doughnuts, how many different options are there for a box of doughnuts?\\
    
    Let $S$ be a multiset $\{6\cdot a,\; 6\cdot b,\; 3\cdot c\}$. Assume $a$ to be chocolate, $b$ to be cinnamon, and $c$ to be plain. Then
    $$|S| = \binom{12+3-1}{12} = 91$$
    Suppose, further, $S'$ be another multiset $\{\infty\cdot a,\;\infty\cdot b,\;\infty\cdot c\}$\\
    
    Similar to the previous question, let's break into several cases.
    \begin{itemize}
        \item Let $A_1$ be a subset where $a$ is selected more than 6 times. Since we assume at least 7 $a$'s are chosen already, we obtain
        $$\binom{5+3-1}{5} = 21$$
        \item Let $A_2$ be a subset where $b$ is selected more than 6 times. Since we assume at least 7 $b$'s are chosen already, we obtain
        $$\binom{5+3-1}{5} = 21$$
        \item Let $A_3$ be a subset where $c$ is selected more than 3 times. Since we assume at least 4 $c$'s are chosen already, we obtain
        $$\binom{8+3-1}{8} = 45$$\\
        
        \item Pairs
        \begin{itemize}
            \item $|A_1\cap A_2| = 0$ (impossible)
            \item $|A_1\cap A_3| = |A_2\cap A_3| = \binom{12-7-4+3-1}{12-7-4} = 3$\\
        \end{itemize}
        
        \item Triple
        \begin{itemize}
            \item Since we already have shown that $|A_1\cap A_2| = 0$, $|A_1\cap A_2\cap A_3| = 0$
        \end{itemize}
    \end{itemize}

    \newpage
    \item[\bf 6.7.9] Determine the number of integral solutions of the equation
    $$x_1+x_2+x_3+x_4 = 20$$ that satisfy
    $$1\le x_1\le 6,\; 0\le x_2\le 7,\; 4\le x_3\le 8,\; 2\le x_4\le 6$$\\
    
    This question is also basically same question as previous two questions. Let 
    \begin{tasks}[style=itemize, column-sep=-35mm, label-align=left, label-offset={0mm}, label-width={3mm}, item-indent={40mm}](2)%
        \task $y_1 = x_1-1$
        \task $y_2 = x_2$
        \task $y_3 = x_3-4$
        \task $y_4 = x_4-2$
    \end{tasks}
    
    \vspace{1.5\baselineskip}
    So that we could make these inequalities in the form of $0\le y_i$. Plug those into the equation and inequalities. Then we obtain $$y_1+y_2+y_3+y_4 = 13$$
    \begin{tasks}[style=itemize, column-sep=-35mm, label-align=left, label-offset={0mm}, label-width={3mm}, item-indent={40mm}](2)%
        \task $0\le y_1\le 5$
        \task $0\le y_2\le 7$
        \task $0\le y_3\le 4$
        \task $0\le y_4\le 4$
    \end{tasks}

    Now, let $S$ be the set of all possible combinations of the equation with regards to $y_i$. Then 
    $$|S| = \binom{13+4-1}{13} = 560$$
    
    `\begin{itemize}
        \item Let $A_1$ be a subset where $a$ is selected more than 5 times. Since we assume at least 6 $a$'s are chosen already, we obtain
        $$\binom{7+3-1}{7} = 120$$
        
         \item Let $A_2$ be a subset where $a$ is selected more than 7 times. Since we assume at least 8 $a$'s are chosen already, we obtain
        $$\binom{5+3-1}{5} = 56$$
        
         \item Let $A_3$ be a subset where $a$ is selected more than 4 times. Since we assume at least 5 $a$'s are chosen already, we obtain
        $$\binom{8+3-1}{8} = 165$$
        
         \item Let $A_4$ be a subset where $a$ is selected more than 4 times. Since we assume at least 5 $a$'s are chosen already, we obtain
        $$\binom{8+3-1}{8} = 165$$
    \end{itemize}
    
    \begin{tasks}[style=itemize, column-sep=0mm, label-align=left, label-offset={0mm}, label-width={3mm}, item-indent={15mm}](2)    
        \task $|A_1\cap A_2| = \binom{13-6-8+3-1}{3-1} = 0$
        \task $|A_1\cap A_3| = \binom{13-6-5+3-1}{3-1} = 10$
        \task $|A_1\cap A_4| = \binom{13-6-5+3-1}{3-1} = 10$
        \task $|A_2\cap A_3| = \binom{13-8-5+3-1}{3-1} = 1$
        \task $|A_2\cap A_4| = \binom{13-8-5+3-1}{3-1} = 1$
        \task $|A_3\cap A_4| = \binom{13-5-5+3-1}{3-1} = 20$\\
    \end{tasks}
    
    Therefore,
    $$|\overline{A_1}\cap\overline{A_2}\cap\overline{A_3}| = 560-(120+56+165+165)+(10+10+1+1+20) = 96$$
    
    \vspace{1.5\baselineskip}
    \item[\bf 6.7.17] Determine the number of permutations of the multiset
    $$S= \{3\cdot a,\;4\cdot b,\;2\cdot c\}$$

    \item[\bf 6.7.31] How many circular permutations are there of the multiset
    $$\{2\cdot a,\;3\cdot b,\;4\cdot c,\;5\cdot d\}$$
    where, for each type of letter, all letters of that type do not appear consecutively?\\
    
    Let $S$ be the set of all permutations of this multiset. Then
    $$|S| = \frac{14!}{2!\cdot 3!\cdot 4!\cdot 5!} = 2522520$$
    
    Consider cases where each kind of letters are deployed consecutively. For these cases, we can consider a series of consecutive letters one big chunk. And let $A_{1,2,3,4}$ be the set of all permutations with all a,b,c,d's consecutive appearing respectively. Then
    \begin{tasks}[style=itemize, column-sep=-52mm, label-align=center, label-offset={0mm}, label-width={3mm}, item-indent={20mm}](2)%    
    \task $|A_1| = \frac{13!}{1!\cdot 3!\cdot 4!\cdot 5!} = 360360$
    \task $|A_2| = \frac{12!}{2!\cdot 1!\cdot 4!\cdot 5!} = 83160$
    \task $|A_3| = \frac{11!}{2!\cdot 3!\cdot 1!\cdot 5!} = 27720$
    \task $|A_4| = \frac{10!}{2!\cdot 3!\cdot 4!\cdot 1!} = 12600$
    \end{tasks}
    
    \begin{tasks}[style=itemize, column-sep=0mm, label-align=left, label-offset={0mm}, label-width={3mm}, item-indent={20mm}](2)   
    \task $|A_1\cap A_2| = \frac{11!}{1!\cdot 1!\cdot 4!\cdot 5!} = 13860$
    \task $|A_1\cap A_3| = \frac{10!}{1!\cdot 3!\cdot 1!\cdot 5!} = 5040$
    \task $|A_1\cap A_4| = \frac{9!}{1!\cdot 3!\cdot 4!\cdot 1!} = 2520$
    \task $|A_2\cap A_3| = \frac{9!}{2!\cdot 1!\cdot 1!\cdot 5!} = 1512$
    \task $|A_2\cap A_4| = \frac{8!}{2!\cdot 1!\cdot 4!\cdot 1!} = 840$
    \task $|A_3\cap A_4| = \frac{7!}{2!\cdot 3!\cdot 1!\cdot 1!} = 420$
    \end{tasks}
    
    \begin{tasks}[style=itemize, column-sep=0mm, label-align=left, label-offset={0mm}, label-width={3mm}, item-indent={12mm}](2)   
    \task $|A_1\cap A_2\cap A_3| = \frac{8!}{1!\cdot 1!\cdot 1\cdot 5!}=336$
    \task $|A_1\cap A_3\cap A_4| = \frac{6!}{1!\cdot 3!\cdot 1\cdot 1!}=120$
    \task $|A_2\cap A_3\cap A_4| = \frac{5!}{2!\cdot 1!\cdot 1\cdot 1!}=60$
    \task $|A_1\cap A_2\cap A_3\cap A_4| = \frac{4!}{1!\cdot 1!\cdot 1\cdot 1!}=24$
    \end{tasks}

\end{enumerate}

\end{document}